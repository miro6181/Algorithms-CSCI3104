\documentclass[12pt]{article}
\setlength{\oddsidemargin}{0in}
\setlength{\evensidemargin}{0in}
\setlength{\textwidth}{6.5in}
\setlength{\parindent}{0in}
\setlength{\parskip}{\baselineskip}
\usepackage{amsmath,amsfonts,amssymb}
\usepackage{graphicx}
\usepackage{float}
\usepackage{enumitem}
\usepackage[]{algorithmicx}
\usepackage{amsthm}
\usepackage{fancyhdr}
\pagestyle{fancy}
\setlength{\headsep}{36pt}
\usepackage{tkz-berge}
\usetikzlibrary{positioning, automata}

\theoremstyle{remark}
\newtheorem*{solution}{Solution}

\usepackage{listings}% http://ctan.org/pkg/listings
\lstset{
  basicstyle=\ttfamily,
  mathescape
}

\usepackage{hyperref}



\newcommand{\makenonemptybox}[2]{%
%\par\nobreak\vspace{\ht\strutbox}\noindent
\item[]
\fbox{% added -2\fboxrule to specified width to avoid overfull hboxes
% and removed the -2\fboxsep from height specification (image not updated)
% because in MWE 2cm is should be height of contents excluding sep and frame
\parbox[c][#1][t]{\dimexpr\linewidth-2\fboxsep-2\fboxrule}{
  \hrule width \hsize height 0pt
  #2
 }%
}%
\par\vspace{\ht\strutbox}
}
\makeatother

\begin{document}
\definecolor {processblue}{cmyk}{0.96,0,0,0}
\lhead{{\bf CSCI 3104, Algorithms \\ Final Exam - 100 pts total} }
\rhead{Name: \fbox{Michael Rogers} \\ ID: \fbox{105667404} \\ {\bf Profs.\ Hoenigman \& Agrawal\\ Fall 2019, CU-Boulder}}
\renewcommand{\headrulewidth}{0.5pt}

\phantom{Test}
\noindent
{\bf \\ Notes:}
%\noindent
\begin{itemize}
\item {\em Due date: 6pm on Sunday, December 8, 2019}
\vspace{-2mm}
\item {\em Submit a pdf file of your written answers to Gradescope and one py file with your Python codes to Canvas. All Python solutions should be clearly commented. Your codes need to run to get credit for your answers.}
\vspace{-2mm}
\item {\em You can ask clarification questions about the exam in office hours and on Piazza. However, please do not ask questions about how to answer a specific question. If there is confusion about any questions, we will address those issues at the beginning of class on December 5.}
\vspace{-2mm}
\item {\em All work on this exam needs to be independent. You may consult the textbook, the lecture notes and homework sets, but you should not use any other resources. If we suspect that you collaborated with anyone in the class or on the Internet, we will enforce the honor code policy strictly. If there is a reasonable doubt about an honor code violation,  you will fail this course.}
\end{itemize}
%\vspace{2mm}

\hrulefill

\begin{enumerate}
\item (10 pts) Consider the following merge() algorithm to merge two sorted arrays into a larger sorted array. There are three errors in the algorithm. 
\begin{verbatim}
MergeWithErrors(A, p, q, r)
    low = A[p..q]
    high = A[q..r]
    i = 0
    j = 0
    k = p
    while(i < q-p+1 and j < r-q)
        if(low[i] <= high[j])
            A[k] = low[i]
            j++
        else
            A[k] = high[j]
            i++
        k++
    while(i < q-p+1)
        A[k] = low[i]
        i++
        k++
    while(j < r-q)
        A[k] = high[j]
        j++
        k++
\end{verbatim}
\begin{enumerate}
    \item (5 pts) List the three errors in the MergeWithErrors algorithm.
    \begin{solution}
    
    \end{solution}
    \pagebreak
    
    \item (5 pts) For the following call to MergeWithErrors, what is the state of the array $A$ after running MergeWithErrors. You can assume that the size of $A$ won't change and values written outside the indices of $A$ will be lost.
    \\ $$A=[0,1,3,5,2,4,6,7]$$
     $$MergeWithErrors(A, 0, 3, 7)$$
    \begin{solution}
    
    \end{solution}
    \pagebreak
    
\end{enumerate}


\item (25 pts) Let $G=(V,E)$ be a directed weighted graph of the pathways on the CU-Boulder campus, with edge weights being distances between different buildings/intersections. Engineering and Humanities are two vertices of $G$, and $k>0$ is a given integer. Assume that you will stop at every building/intersection you pass by. A shortest $k$-stop path is a shortest path between two vertices with exactly $k$ stops.
\begin{enumerate}
    \item (5 pts) Provide an example showing that the shortest $k$-stop path can't necessarily be found using Breadth-first search or Dijkstras algorithm. You need an example for each algorithm that shows where it fails.
    \begin{solution}
    
    \end{solution}
    \pagebreak

    \item (10 pts) Design an algorithm to find the shortest path from Engineering to Humanities that contains exactly $k$ stops (excluding Engineering and Humanities). Notice that a $k$-stop path from these two buildings may not exist. So, your algorithm should also take care of such possibility. You need to provide an explanation of how your algorithm works to receive credit for this question.
    \begin{solution}
    
    \end{solution}
    \pagebreak

    \item (10 pts) Implement your algorithm using the starter code provided on Canvas. 
    
\end{enumerate}

\item (25 pts) To entertain her kids during a recent snowstorm, Dr. Hoenigman invented a card game called EPIC!. In the two-player game, an even number of cards are laid out in a row, face up so that players can see the cards' values. On each card is written a positive integer, and players take turns removing a card from either end of the row and placing the card in their pile. The objective of the game is to collect the fewest points possible. The player whose cards add up to the lowest number after all cards have been selected wins the game.
\\One strategy is to use a greedy approach and simply pick the card at the end that is the smallest. However, this is not always optimal, as the following example shows: (The first player would win if she would first pick the 5 instead of the 4.)

4 2 6 5

\begin{enumerate}
\item (10 pts) Write a non-greedy, efficient and optimal algorithm for a strategy to play EPIC!. The runtime needs to be less than $\theta(n^2)$. Player 1 will use this strategy and Player 2 will use a greedy strategy of choosing the smallest card. \textbf{Note: Your choice of algorithmic strategy really matters here. Think about the types of algorithms we've learned this semester when making your choice.} You need to provide an explanation of how your algorithm works to receive credit for this question.
    \begin{solution}
    
    \end{solution}
    \pagebreak

\item (15 pts) Implement your strategy and the greedy strategy in Python and include code to simulate a game. Your simulation work for up to 100 cards, and values ranging from 1 to 100. Your simulation should include a randomly generated collection of cards and show the sum of cards in each hand at the end of the game. 
\end{enumerate}

\pagebreak
\item (22 pts) 
	In a previous homework assignment and classroom activity, we worked on the problem of finding the peak in an array, where array $A[1, 2, \cdots, n]$ with the property that the subarray $A[1..i]$ has the property that $A[j]>A[j+1]$ for $1\leq j< i$, and the subarray $A[i..n]$ has the property that $A[j]<A[j+1]$ for $i\leq j < n$. For example, $A=[16, 15, 10, 9, 7, 3, 6, 8, 17, 23]$ is a peaked array.\\
	
	Now consider the \textit{multi-peaked} generalization, in which the array contains $k$ peaks, i.e., it contains $k$ subarrays, each of which is itself a peaked array. Suppose that $k=2$ and we can guarantee that neither peak is closer than $n/4$ positions to the middle of the array, and that the ``joining point'' of the two singly-peaked subarrays lays in the middle half of the array. 
	\begin{enumerate}
	    \item (8 pts) Now write an algorithm that returns the minimum element of $A$ in sublinear time.
	    \begin{solution}
    
        \end{solution}
        \pagebreak

	    \item (10 pts) Prove that your algorithm is correct. (Hint:\ prove that your algorithm's correctness follows from the correctness of another correct algorithm we already know.)
	    \begin{solution}
    
        \end{solution}
        \pagebreak

	    \item (4 pts) Give a recurrence relation for its running time, and solve for its asymptotic behavior.
	   \begin{solution}
    
        \end{solution}
        \pagebreak

	\end{enumerate}

\item (18 pts) Suppose we are given a set of non-negative distinct numbers $S$ and a target $t$. We want to find if there exists a subset of $S$ that sums up to \textbf{exactly} $t$ and the cardinality of this subset is $k$. 
\\Write a python program that takes an input array $S$, target $t$, and cardinality $k$, and returns the subset with cardinality $k$ that adds to $t$ if it exists, and returns $False$ otherwise. Your algorithm needs to run in $O(nt)$ time where $t$ is the target and $n$ is the cardinality of $S$. In your code, provide a brief discussion of your runtime through comments, referring to specific elements of your code.\\
\\For example - 

Input: s = \{2,1,5,7\}, t = 4, k = 2\\
Output: False\\
Explanation:  No subset of size $2$ sums to 4.\\\\
Input: s = \{2,1,5,7\}, t = 6, k = 2\\
Output: \{1, 5\}\\
Explanation:  Subset \{1, 5\} has size 2 and sums up to the target $t = 6$.\\\\
Input: s = \{2,1,5,7\}, t = 6, k = 3\\
Output: False\\
Explanation:  No subset of size $3$ sums to 6.\\\\

\noindent
\end{enumerate}





\end{document}

