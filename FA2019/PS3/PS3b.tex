\documentclass[12pt]{article}
\setlength{\oddsidemargin}{0in}
\setlength{\evensidemargin}{0in}
\setlength{\textwidth}{6.5in}
\setlength{\parindent}{0in}
\setlength{\parskip}{\baselineskip}
\usepackage{amsmath,amsfonts,amssymb}
\usepackage{graphicx}
\usepackage{enumitem}
\usepackage[]{algorithmicx}
\usepackage{amsthm}
\usepackage{fancyhdr}
\pagestyle{fancy}
\setlength{\headsep}{36pt}
\usepackage{tkz-berge}
\usetikzlibrary{positioning, automata}

\usepackage{hyperref}

\theoremstyle{remark}
\newtheorem*{solution}{Solution}

\newcommand{\makenonemptybox}[2]{%
%\par\nobreak\vspace{\ht\strutbox}\noindent
\item[]
\fbox{% added -2\fboxrule to specified width to avoid overfull hboxes
% and removed the -2\fboxsep from height specification (image not updated)
% because in MWE 2cm is should be height of contents excluding sep and frame
\parbox[c][#1][t]{\dimexpr\linewidth-2\fboxsep-2\fboxrule}{
  \hrule width \hsize height 0pt
  #2
 }%
}%
\par\vspace{\ht\strutbox}
}
\makeatother

\begin{document}
\definecolor {processblue}{cmyk}{0.96,0,0,0}

\lhead{{\bf CSCI 3104, Algorithms \\ Problem Set 3b (50 points)} }
\rhead{Name: \fbox{Michael Rogers} \\ ID: \fbox{105667404} \\ {\bf Profs.\ Hoenigman \& Agrawal\\ Fall 2019, CU-Boulder}}
\renewcommand{\headrulewidth}{0.5pt}

\phantom{Test}

\begin{small}
\textbf{Instructions for submitting your solution}:
\vspace{-5mm} 

\begin{itemize}
	\item The solutions \textbf{should be typed} and we cannot accept hand-written solutions. \href{http://ece.uprm.edu/~caceros/latex/introduction.pdf}{Here's a short intro to Latex.}
	\item You should submit your work through \href{https://www.gradescope.com/courses/59294}{\textbf{Gradescope}} only.
	\item If you don't have an account on it, sign up for one using your CU email. You should have gotten an email to sign up. If your name based CU email doesn't work, try the identikey@colorado.edu version. 
	\item Gradescope will only accept \textbf{.pdf} files (except for code files that should be submitted separately on Gradescope if a problem set has them) and \textbf{try to fit your work in the box provided}. 
	\item You cannot submit a pdf which has less pages than what we provided you as Gradescope won't allow it. 
	\item Verbal reasoning is typically insufficient for full credit. Instead, write a logical argument, in the style of a mathematical proof.
	\item For every problem in this class, you must justify your answer:\ show how you arrived at it and why it is correct. If there are assumptions you need to make along the way, state those clearly.
	
	\item You may work with other students. However, \textbf{all solutions must be written independently and in your own words.} Referencing solutions of any sort is strictly prohibited. You must explicitly cite any sources, as well as any collaborators. 
\end{itemize}



\vspace{-4mm} 
\end{small}

\hrulefill

\newpage
\begin{enumerate}

\item (23 pts) Imagine an alternate reality where CU has a small robot that travels around the campus delivering food to hungry students. The robot starts at the C4C and goes to whatever dorm or classroom has placed the order. The fully-charged battery of the robot has enough energy to travel $k$ meters. On campus, there are $n$ wireless charging pods where the robot can stop to charge its battery. Denote by $l_1 < l_2 < \cdots < l_n$  the locations of the charging pods along the route with $l_i$ the distance from the C4C to the $i${\em th} charging pod. The distance between neighboring charging pods is assumed to be at most $k$ meters. Your objective is to make as few charging stops as possible along the way. 
\begin{enumerate}
\item (10 pts) Write a python program for an optimal greedy algorithm to determine at which charging pods the robot would stop. Your code should take as input $k$ and a \textit{list} of distances of charging pods (first distance in the list is 0 to represent the start point and the last is the destination and not a pod). Print out the charging pods where the robot stops using your greedy strategy.\\
Example 1 - If \textbf{k = 40} and \textbf{Pods = [0, 20, 37, 54, 70, 90]}. Number of stops required is 2 and the output should be \textbf{[37, 70]}.\\
Example 2 - If \textbf{k = 20} and \textbf{Pods = [0, 18, 21, 24, 37, 56, 66]}. Number of stops required is 3 and the output should be \textbf{[18, 37, 56]}.\\
Example 3 - If \textbf{k = 20} and \textbf{Pods = [0, 10, 15, 18]}. Number of stops required is 0 and the output should be \textbf{[]}.\\



% Give an optimal greedy algorithm to determine at which charging stations you would stop. 
\item (3 pts) Provide the time complexity of your python algorithm, including an explanation.
\pagebreak
\item (10 pts) Prove that your algorithm gives an optimal solution.
\end{enumerate}
\pagebreak

\item (7 pts) Using Dijkstra's algorithm, determine the length of the shortest path from $v_{0}$ to each of the other vertices in the graph. Clearly specify the distances from $v_{0}$ to each vertex \textbf{after each iteration} of the algorithm.
\begin{center}
	\begin {tikzpicture}[-latex ,auto ,node distance =2 cm and 3cm ,on grid ,
	semithick ,
	state/.style ={ circle ,top color =white , bottom color = processblue!20 ,
	draw,processblue , text=blue , minimum width =1 cm}]

	\node[state] (A) {$v_{0}$};
	\node[state] (B) [above right = of A] {$v_{1}$};
	\node[state] (C) [below right = of A] {$v_{2}$};
	\node[state] (D) [right = of B] {$v_{3}$};
	\node[state] (E) [right = of C] {$v_{4}$};
	\node[state] (F) [below right = of D] {$v_{5}$};
	\path (A) edge node[above] {$2$} (B);
	\path (A) edge node[right] {$7$} (C);
	\path (B) edge node[right] {$4$} (C);
	\path (B) edge node[above] {$3$} (D);
	\path (B) edge node[right] {$1$} (E);
	\path (C) edge node[above] {$6$} (E);
	\path (E) edge node[right] {$1$} (D);
	\path (D) edge node[above] {$13$} (F);
	\path (E) edge node[above] {$19$} (F);

	\path (C) edge [bend right = 50] node[below] {$3$} (F);
	
	\end{tikzpicture}  
\end{center}
\pagebreak


\item (20 pts) After years of futility, the Colorado Rockies have decided to try a new approach to signing players. Next year, they have a target number of wins, $n$, and they want to sign the fewest number of players who can produce exactly those $n$ wins. In this model, each player has a win value of $v_{1}<v_{2}<\dots<v_{r}$ for $r$ player types, where each player's value $v_{i}$ is a positive integer representing the number of wins he brings to the team. (Note: In a real-world example, All-Star third baseman, Nolan Arenado, contributed 4.5 wins this year beyond what a league-minimum player would have contributed to the team.) 
The team's goal is to obtain a set of counts $\{d_{i}\}$, one for each player type (so $d_{i}$ represents the quantity of players with valuation $v_{i}$ that are recruited), such that $\sum_{i=1}^{r}d_{i}=k$ and where $k$ is the number of players signed, and $k$ is minimized.
	\begin{enumerate}
	\item  (10 pts) Write a greedy algorithm that will produce an optimal solution for a set of player win values of $[1, 2, 4, 8, 16]$ and prove that your algorithm is optimal for those values. Your algorithm need only be optimal for the fixed win values $[1, 2, 4, 8, 16]$. You do \textbf{not} need to consider other configuration of win values.
	\begin{solution}

    \end{solution}
    \pagebreak

[Additional space for solving Q3a]
\pagebreak

\item (10 pts) Find a set of win values where your algorithm does not produce the optimal solution and show where your algorithm fails for those values. 
\begin{solution}

\end{solution}

    \end{enumerate}

\end{enumerate}

\pagebreak
\textbf{Ungraded questions} - These questions are for your practice. We won't grade them or provide a typed solution but we are open to discuss these in our OHs and you should take feed backs on your approach during the OHs. These questions are part of the syllabus. 

\begin{enumerate}

\item Suppose we have a directed graph $G$, where each edge $e_{i}$ has a weight $w_{i} \in (0, 1).$  The weight of a path is the product of the weights of each edge. 
\begin{enumerate}[label=(\alph*)]
\item Explain why a version of Dijkstra's algorithm cannot be used here. [\textbf{Hint:} We may think about transforming $G$ into a graph $H$, where the weight of edge $i$ in $H$ is $\ln(w_{i})$. It is equivalent to apply Dijkstra's algorithm to $H$.]
\begin{solution}

\end{solution}

\item What conditions does each edge weight $w_{i}$ need to satisfy, in order to use Dijkstra's algorithm?
\begin{solution}

\end{solution}


\end{enumerate}



\end{enumerate}

\end{document}
