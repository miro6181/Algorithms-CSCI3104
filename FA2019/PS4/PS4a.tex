\documentclass[12pt]{article}
\setlength{\oddsidemargin}{0in}
\setlength{\evensidemargin}{0in}
\setlength{\textwidth}{6.5in}
\setlength{\parindent}{0in}
\setlength{\parskip}{\baselineskip}
\usepackage{amsmath,amsfonts,amssymb}
\usepackage{graphicx}
\usepackage{enumitem}
\usepackage[]{algorithmicx}
\usepackage{amsthm}
\usepackage{fancyhdr}
\pagestyle{fancy}
\setlength{\headsep}{36pt}

\usepackage{hyperref}

\theoremstyle{remark}
\newtheorem*{solution}{Solution}

\newcommand{\makenonemptybox}[2]{%
%\par\nobreak\vspace{\ht\strutbox}\noindent
\item[]
\fbox{% added -2\fboxrule to specified width to avoid overfull hboxes
% and removed the -2\fboxsep from height specification (image not updated)
% because in MWE 2cm is should be height of contents excluding sep and frame
\parbox[c][#1][t]{\dimexpr\linewidth-2\fboxsep-2\fboxrule}{
  \hrule width \hsize height 0pt
  #2
 }%
}%
\par\vspace{\ht\strutbox}
}
\makeatother

\begin{document}

\lhead{{\bf CSCI 3104, Algorithms \\ Problem Set 4a (11 points)} }
\rhead{Name: \fbox{Michael Rogers} \\ ID: \fbox{105667404} \\ {\bf Profs.\ Hoenigman \& Agrawal\\ Fall 2019, CU-Boulder}}
\renewcommand{\headrulewidth}{0.5pt}

\phantom{Test}

\begin{small}
\textbf{Instructions for submitting your solution}:
\vspace{-5mm} 

\begin{itemize}
	\item The solutions \textbf{should be typed} and we cannot accept hand-written solutions. \href{http://ece.uprm.edu/~caceros/latex/introduction.pdf}{Here's a short intro to Latex.}
	\item You should submit your work through \href{https://www.gradescope.com/courses/59294}{\textbf{Gradescope}} only.
	\item If you don't have an account on it, sign up for one using your CU email. You should have gotten an email to sign up. If your name based CU email doesn't work, try the identikey@colorado.edu version. 
	\item Gradescope will only accept \textbf{.pdf} files (except for code files that should be submitted separately on Gradescope if a problem set has them) and \textbf{try to fit your work in the box provided}. 
	\item You cannot submit a pdf which has less pages than what we provided you as Gradescope won't allow it. 
	\item Verbal reasoning is typically insufficient for full credit. Instead, write a logical argument, in the style of a mathematical proof.
	\item For every problem in this class, you must justify your answer:\ show how you arrived at it and why it is correct. If there are assumptions you need to make along the way, state those clearly.
	
	\item You may work with other students. However, \textbf{all solutions must be written independently and in your own words.} Referencing solutions of any sort is strictly prohibited. You must explicitly cite any sources, as well as any collaborators. 
\end{itemize}
\vspace{-4mm} 
\end{small}

\hrulefill

\newpage
\begin{enumerate}
\item (1 pt) What is the definition of a Minimum Spanning Tree (MST)?
\begin{solution}

\end{solution}
\item (1 pt) Describe in one or two sentences, a greedy rule for constructing an MST.
\begin{solution}

\end{solution}

\item (3 pts) How many unique MSTs does the following graph have :
\begin{figure}[h!]
\begin{center}
\includegraphics[scale=0.3]{mst_graph_q2.jpg} 
\end{center}
\end{figure}


\begin{solution}

\end{solution}

\pagebreak
\item (3 pts) Suppose that you have calculated the MST of an undirected graph $G=(V,E)$ with positive edge weights. \\
If you increase each edge weight by 2, will the MST change? Prove that it cannot change or give a counterexample if it changes. (Note: Your proof, if there is one, can be a simple logical argument.)
\begin{solution}


\end{solution}
\pagebreak

\item (3 pts) Suppose that you have calculated the shortest paths to all vertices from a fixed vertex $s\in V$ of an undirected graph $G=(V,E)$ with positive edge weights. \\
If you increase each edge weight by 2, will the shortest paths from $s$ change? Prove that it cannot change or give a counterexample if it changes.(Note: Just as in Part a, your proof can be a simple logical argument.)
\begin{solution}

\end{solution}

\pagebreak


\end{enumerate}

\end{document}


