\documentclass[12pt]{article}
\setlength{\oddsidemargin}{0in}
\setlength{\evensidemargin}{0in}
\setlength{\textwidth}{6.5in}
\setlength{\parindent}{0in}
\setlength{\parskip}{\baselineskip}
\usepackage{amsmath,amsfonts,amssymb}
\usepackage{graphicx}
\usepackage{enumitem}
\usepackage[]{algorithmicx}
\usepackage{amsthm}
\usepackage{fancyhdr}
\pagestyle{fancy}
\setlength{\headsep}{36pt}
\usepackage{tkz-berge}
\usetikzlibrary{positioning, automata}

\usepackage{hyperref}

\theoremstyle{remark}
\newtheorem*{solution}{Solution}

\newcommand{\makenonemptybox}[2]{%
%\par\nobreak\vspace{\ht\strutbox}\noindent
\item[]
\fbox{% added -2\fboxrule to specified width to avoid overfull hboxes
% and removed the -2\fboxsep from height specification (image not updated)
% because in MWE 2cm is should be height of contents excluding sep and frame
\parbox[c][#1][t]{\dimexpr\linewidth-2\fboxsep-2\fboxrule}{
  \hrule width \hsize height 0pt
  #2
 }%
}%
\par\vspace{\ht\strutbox}
}
\makeatother

\begin{document}

\lhead{{\bf CSCI 3104, Algorithms \\ Problem Set 10b (34 points)} }
\rhead{Name: \fbox{Michael Rogers} \\ ID: \fbox{105667404} \\ {\bf Profs.\ Hoenigman \& Agrawal\\ Fall 2019, CU-Boulder}}
\renewcommand{\headrulewidth}{0.5pt}

\phantom{Test}

\begin{small}
\textit{Advice 1}:\ For every problem in this class, you must justify your answer:\ show how you arrived at it and why it is correct. If there are assumptions you need to make along the way, state those clearly.
\vspace{-3mm} 

\textit{Advice 2}:\ Verbal reasoning is typically insufficient for full credit. Instead, write a logical argument, in the style of a mathematical proof.\\
\vspace{-3mm} 

\textbf{Instructions for submitting your solution}:
\vspace{-5mm} 

\begin{itemize}
	\item The solutions \textbf{should be typed} and we cannot accept hand-written solutions. \href{http://ece.uprm.edu/~caceros/latex/introduction.pdf}{Here's a short intro to Latex.}
	\item You should submit your work through \href{https://www.gradescope.com/courses/59294}{\textbf{Gradescope}} only.
	\item If you don't have an account on it, sign up for one using your CU email. You should have gotten an email to sign up. If your name based CU email doesn't work, try the identikey@colorado.edu version. 
	\item Gradescope will only accept \textbf{.pdf} files (except for code files that should be submitted separately on Gradescope if a problem set has them) and \textbf{try to fit your work in the box provided}. 
	\item You cannot submit a pdf which has less pages than what we provided you as Gradescope won't allow it. 
\end{itemize}
\vspace{-4mm} 
\end{small}

\hrulefill
\pagebreak
\\
\textbf{Important:} This assignment has 1 (Q1) coding question. 
\begin{itemize}
    \item You need to submit 1 python file.
    \item The .py file should run for you to get points and name the file as following - \\
If Q1 asks for a python code, please submit it with the following naming convention - \texttt{Lastname-Firstname-PS10b-Q1.py}.
\item You need to submit the code via Canvas but the table/plot/result should be on the main .pdf.
\end{itemize}
\pagebreak

\begin{enumerate}
    
    	\item (34 pts total) Recall that the \textit{string alignment problem} takes as input two strings $x$ and $y$, composed of symbols $x_{i},y_{j}\in \Sigma$, for a fixed symbol set $\Sigma$, and returns a minimal-cost set of \textit{edit} operations for transforming the string $x$ into string $y$.
	
	Let $x$ contain $n_{x}$ symbols, let $y$ contain $n_{y}$ symbols, and let the set of edit operations be those defined in the lecture notes (substitution, insertion, and deletion).
	
	Let the cost of \textit{insert} be $c_{insert}$ and \textit{delete} be $c_{delete}$, and the cost of \textit{sub} be $c_{sub}$, except when $x_{i}=y_{j}$, which is a ``no-op'' and has cost 0.
	
	In this problem, you will implement and apply three functions.
	
	(i) {\tt alignStrings(x,y, $c_{insert}$, $c_{delete}$, $c_{sub}$)} takes as input two ASCII strings $x$ and $y$, cost of the operations, and runs a dynamic programming algorithm to return the cost matrix $S$, which contains the optimal costs for all the subproblems for aligning these two strings. 
	
	
	(ii) {\tt extractAlignment(S,x,y, $c_{insert}$, $c_{delete}$, $c_{sub}$)} takes as input an optimal cost matrix $S$, strings $x,y$, cost of the operations, and returns a vector $a$ that represents an optimal sequence of edit operations to convert $x$ into $y$. This optimal sequence is recovered by finding a path on the implicit DAG of decisions made by {\tt alignStrings} to obtain the value $S[n_{x},n_{y}]$, starting from $S[0,0]$. 
	
	When storing the sequence of edit operations in $a$, use a special symbol to denote no-ops.
	


	(iii) {\tt commonSubstrings(x,L,a)} which takes as input the ASCII string $x$, an integer $1\leq L \leq n_{x}$, and an optimal sequence $a$ of edits to $x$, which would transform $x$ into $y$. This function returns each of the substrings of length at least $L$ in $x$ that aligns exactly, via a run of no-ops, to a substring in $y$.
		\begin{enumerate}

	\item (21 pts) From scratch, implement the functions {\tt alignStrings}, {\tt extractAlignment}, and {\tt commonSubstrings}. You may not use any library functions that make their implementation trivial. Within your implementation of {\tt extractAlignment}, ties must be broken uniformly at random.\\
	If you plan to create a version that saves the parent information in {\tt alignStrings} itself, then you should break the ties randomly in {\tt alignStrings} instead. 
	
	\textbf{Submit: }
	\begin{itemize}
	    \item A brief paragraph for each function that explains how you implemented it (describe how it works and how it uses its data structures).
	    \item Your code implementation, with code (the code should be submitted on Canvas)
	    \item The cost matrix $S$ that your code produces on the strings  x={\tt EXPONENTIAL} and y={\tt POLYNOMIAL} with $c_{insert} = 2$, $c_{delete} = 1$, $c_{sub} = 2$
	\end{itemize}
	\label{q:align:code}
	
	\begin{solution}
	
	
	\end{solution}
	
	
	\item (7 pts) Using asymptotic analysis, determine the running time of the call \\ ${}^{}$\hspace{0mm} {\tt commonSubstrings(x, L, extractAlignment(  alignStrings(x,y,$c_{insert}$,$c_{delete}$,$c_{sub}$), x,y,$c_{insert}$,$c_{delete}$,$c_{sub}$  )  )} \\
	Justify your answer.

	\begin{solution}
	
	
	\end{solution}



	\item (6 pts) \textbf{Plagiarism detector} - String alignment algorithms can be used to detect changes between different versions of the same document (as in version control systems) or to detect verbatim copying between different documents (as in plagiarism detection systems).
	
	The two {song lyrics} files for PS10b (see class Canvas) contain lyrics of two different songs in text format. Use your functions from~\eqref{q:align:code} with $c_{insert} = 1$, $c_{delete} = 1$, $c_{sub} = 1$ to align the text of these two documents. Utilize your \textbf{commonSubstrings} function for plagiarism detection. Present the results of your analysis, including a reporting of all the substrings in $x$ of length $L=10$ or more that could have been taken from $y$ in two columns with the first being the length of the substring and the second being the actual common substring obtained via continuous 'no-op' run.  \\Briefly comment on whether these songs could be reasonably considered original works, under CU's academic integrity policy.
	
	
	\begin{solution}
	
	
	\end{solution}
	
	\end{enumerate}
\end{enumerate}


\end{document}


